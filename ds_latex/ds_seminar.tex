%%% Angepasste Version der unitext Vorlage von Dr. Werner Struckmann (https://www.tu-braunschweig.de/ips/staff/struckmann/unitext)
%%% für wissenschaftliche Arbeiten am Institut für Wirtschaftsinformatik, Abteilung Decision Support

%%% Die Klasse unitext besitzt die Optionen
%%% bachelorarbeit, projektarbeit, masterarbeit, studienarbeit, diplomarbeit,
%%% bericht und script für die verschiedenen Dokumenttypen. 
%%% Bei Bedarf k?nnen weitere aufgenommen werden. Aus jeder der beiden Gruppen ist
%%% genau eine Option zu wählen. Außerdem können der Stil des Literaturverzeich-
%%% nisses (alpha, abbrv, unsrt, plain, apalike) und die Sprache (german, english) als
%%% weitere Optionen angegeben werden. Die Angabe einer Sprache ist obligato-
%%% risch. Eine der Optionen dvi, ps oder pdf ist abhängig vom gewählten Ausgabe-
%%% format anzugeben. Dokumentspezifische Einstellungen werden in unitext.cfg
%%% vorgenommen. Der Text ist doppelseitig auszudrucken.
%%% Weitere Optionen: Beispiele für Titelseite: titelseite, cd

% pdf durch Ausgabeformat ersetzen: dvi, ps  oder pdf.

\documentclass[ds,english,bachelorarbeit,pdf,cd,apacite]{unitext} 

%\documentclass[ds,english,seminar,pdf,cd,apacite]{unitext}                % Default

        % Beispiel 1
% \documentclass[ds,german,masterarbeit,pdf,cd,apacite]{unitext}         % Beispiel 2

\usepackage{amsmath}
\DeclareMathOperator*{\argmin}{arg\,min}
\usepackage{graphicx}
\usepackage{subfigure}
\usepackage{enumitem}
\setlist[description]{leftmargin=\parindent,labelindent=\parindent}
\usepackage{hyperref}

\usepackage{setspace}
\usepackage[section]{placeins}
\onehalfspacing
%%% title, author und date m?ssen angegeben werden,
%%% dozent, betreuer und keywords sind optional.
\title{Comparison of Supervised Learning Methods for Arrival Time Estimation in Meal Delivery}
\author{Emre Gezer}
\matrikelnummer{4901507}
\studiengang{Wirtschaftsinformatik}

\dozent{Jun.-Prof. Dr. Marlin Ulmer}
\betreuer{M. Sc. Florentin D. Hildebrandt}

\date{10.02.2021} %%% Abgabedatum

\keywords{Wirtschaftsinformatik, Decision Support}


\makeglossary           %%% optional
\makeindex              %%% optional

\begin{document}

%%% Titelteil
\titelblatt             %%% obligatorisch
\tableofcontents        %%% obligatorisch
\erklaerung             %%% obligatorisch fuer Abschlussarbeiten
%listoftables           %%% optional
\listoffigures          %%% optional
%\abkuerzung             %%% optional, in der Datei abkuerzung.tex


%%% ------------- Der Textteil -------------
\starttext              

\chapter{Introduction}
%%% Motiviere Problem
Nobody likes waiting for food - or waiting at all. In fact, when doing research on this topic, psychologists found out that increased waiting times generally have a significant negative impact on customer satisfaction and loyalty (\citealt{WaitingTime1}, \citealt{WaitingTime2}). Combine this with the fact that the online food delivery market is in high demand as more than 700 million people globally used food delivery services in 2017 and twice as many expected in 2024 \citep{Statista1}. In the US, the biggest platform-to-delivery companies like GrubHub, Domino's Pizza and UberEATS have close market shares. Thus, waiting not only becomes a large scaled economic issue due to the already very big user base. The participants on the provider side, be it the numerous restaurants or the meal delivery platforms, also operate in highly competitive environments. However, one might not expect that the customers’ own perceptions regarding their waiting time negatively affect the perceived service quality stronger than actual waiting times do (\citealt{waiting5}, \citealt{waiting6}). Given these points, we can conclude that a key challenge, but also a chance lies in the communication of accurate arrival time estimations to customers. 
%%% Motiviere Lösung

A popular and common way to tackle prediction tasks in general is the use of machine learning techniques. In the last years, it has proven to be a powerful forecasting tool in a wide variety of settings. For arrival time estimation, a go-to choice seems to be the use of supervised learning: Exemplary, Hildebrandt and Ulmer use Gradient Boosting Decision Trees in their supervised learning approach to map state features directly to expected arrival times in food delivery. However, different researchers use different approaches on similar problem settings: In contrast to \cite{Hildebrandt2020_EAT}, \cite{Zhu2020_OFCTE_DL} use deep learning to predict accurate order fulfillment cycle times in a meal delivery environment. Moreover, both propose different data models their proposed supervised model learns upon. This is not an isolated case. As we will show later on, such instances are rather mounting up. 
To yield optimal results, not only the choice of the right data and the right learning model is crucial, but moreover a suitably chosen parametrization of the learning model as well. The art of optimizing model parameters is a highly active research area as hyperparameter has proven that it can improve the performance of models in different environments (\citealt{HPOMotivation}, \citealt{WU201926}).
Drawing the connection to the highly competetive market that online food delivery certainly is, we can conclude that choosing a model that yields the best possible results is a significant competitive advantage for any meal delivery. For that reason, we intend to conduct further research in this area.

This paper examines the forecast quality of different offline supervised learning models for arrival time estimation problems in meal delivery from different view points. It builds up on the work of \cite{Hildebrandt2020_EAT}, who formulate the \textit{Restaurant Meal Delivery Problem with Arrival Times}. The RMDPEAT combines arrival time estimation with delivery routing. The underlying vehicle routing setting for the RMDPEAT originates from the \textit{Restaurant Meal Delivery Problem}, a dynamic pick-up and delivery problem originally presented in \cite{UlmerRMDP}. We operate on the same problem setting as \cite{Hildebrandt2020_EAT} and will therefore also operate on similar data. The dataset contains historical meal delivery data collected within Iowa City.  

This paper is organized as follows. Firstly, we will review and discuss literature that focuses on arrival time estimations done via offline supervised learning methods in chapter \ref{chap:review}. We will then proceed with the problem formulation in \ref{chap:prob}. There, we will recap the RMDP originally presented in \cite{UlmerRMDP}, and work out the differences to the RMDPEAT presented in \cite{Hildebrandt2020_EAT}. By that, we intend to motivate the integration and use of arrival time estimation in the RMDP. In chapter \ref{chap:method}, we first present the design of our experimental pipeline consisting of three different experiments which can be understood as the main conrtibutions of this paper:
\begin{enumerate}
	\item Analyze the model performance behaviour for subsets of different size in order to capture the sufficient sample size needed to train the respective model.
	\item Conduct hyperparameter optimization to attain the optimal configurations and further analyze in what way the parameters of the model impact the outcome. 
	\item Examine the robustness of the models by training the models with different levels of noise in the data.
\end{enumerate}
Then, we present and explain our data model selection, and last but not least work out the supervised learning methods used for arrival time prediction in detail. We will then advance to the computational study in chapter \ref{chap:comp}, where we present and discuss the results of our experiments.
Chapter \ref{chap:conc} draws the conclusions and gives outlook on future work.
 
\chapter{Literature Review}

In this section, the paper presents related literature. 


\section{Most related work}

This work is mostly inspired by \cite{Hildebrandt2020_EAT}, who contributed a supervised offline and another supervised online-offline approach to predict arrival times in the Restaurant Meal Delivery problem setting with estimation of arrival time (RMDPEAT), a dynamic pick up and delivery problem with uncertainty in travel times, processing times and order bundling firstly  presented in \cite{UlmerRMDP}. A more detailed problem explanation of the RMDPEAT is done in Section 3. 
In their offline approach, \cite{Hildebrandt2020_EAT} map spatial, temporal, routing, and processing features based on the RMDPEAT to expected arrival times by means of a gradient boosting decision tree (GBDT) model. In their online-offline approach, they calculate the running arrival time average online by training a supervised neural network that approximates the decisions a given routing policy would make for a given state.
This paper can be seen as complementary to their paper since we aim to estimate arrival times based on the same underlying problem setting via several supervised learning algorithms, including GBDTs. However, due to hardware limitations, partial and full online approaches are excluded. 

\cite{Zhu2020_OFCTE_DL} predict arrival times by means of deep learning for a vehicle routing problem where requests come in dynamically. Uncertainty is present in bundling, courier travel times, courier waiting times at restaurants and cooking times. Additional information 

\section{Arrival Time Estimation}

\cite{UlmerBarrett2017_TWAP} estimate mean arrival times for a vehicle routing problem with stochastic requests (VRPSR) in home-attended delivery using value function approximation. While their model assumes uncertainty in travel 



\cite{Liu2018_LM_PLM} ETA x Machine Learning x RMDP
 
   


\cite{Zhang2013} formulate a SVRPTW with soft-time window constraints and stochastic travel and service times. Their objective function seeks to minimize vehicle employing costs, travel times and deviances w.r.t expected arrival times which are determined by the sum of expected service and travel times. The expected arrival times are then used to calculate a minimum on-time probability which every vehicle must fulfill for every visited customer.  

\cite{Wu2004_SVR} apply support vector regression (SVR) for travel time prediction for single origin-destination routes in the context of intelligent transportation systems. Their experimental procedure contains of selecting a relatively less biased subset of the provided traffic data, and comparing the SVR to two analytical baseline methods for travel time prediction as benchmarks. They rely solely on temporal features.

\cite{Vanajakshi2007_SVR}
\cite{Masiero2011_SVR}

\cite{Chen2004_ANNKalman} who proposed an artifical neural network.

\cite{Siripanpornchana2016_AnnWithDbnFS}

\cite{Goudarzi2018Comparison}

\cite{Wang2018_WDR_DL} propose an offline and online method both based on a Wide Deep Recurrent Neural Network (WDRNN) model to predict vehicle travel times for single origin-destination routes by aggregating spatial, temporal, traffic, personalized and augmented features. In their offline comparison, they compared their approach to competing classical machine learning methods and route-based ETA. Two indications of their results are interesting in our viewpoint: First, all machine learning methods included in their experiment outperformed route-based ETA. Secondly, their deep learning approach outperformed the competing classical machine learning methods.  

\cite{Cheng2019_GBDT} predict travel times to improve the accuracy of freeway traffic flow prediction by means of a GBDT model. Their results indicate that GBDTs outperform feedforward neural networks and support vector machines. 
 
\cite{Huang2018_GBDT} and \cite{huang2020travel_GBDT} compare several tree-based learning methods for taxi travel time prediction on different horizons each, among them random forests (both), GBDTs (both), and CART (only \cite{huang2020travel_GBDT}). While \cite{Huang2018_GBDT} selected features by means of principal component analysis, \cite{huang2020travel_GBDT} engineered them manually. Both ended up using spatial and temporal features mainly. Their results indicate that all tree-based ensemble methods are able to predict travel times more accurately than the respective benchmark algorithms (CART and naive approach in \cite{huang2020travel_GBDT}; linear and logistic regression in \cite{Huang2018_GBDT}).
In contrast to them, \cite{jindal2017unified} estimate taxi travel times with a unified approach based on neural networks. Distances for a taxi trips predicted based on raw spatial data are used to predict travel times based on raw spatiotemporal data.

\chapter{Problem formulation}

This chapter gives a short overview of the RMDPEAT introduced in \cite{Hildebrandt2020_EAT}. The RMDPEAT combines the dynamic pick-up and delivery problem presented in \cite{UlmerRMDP} with arrival time estimation and hence accounts for both processes. The integration of these fields is important since it opens up the possibility for customers to choose a preferred restaurant or none at all based on estimated arrival times. We will use the route based markov decision process model notation.











	
    


\chapter{Solution Approach}
This chapter presents the solution approach.
Section 4.1 motivates our problem. 
Section 4.2 discusses the selected features.
Section 4.3 explains the algorithms considered in the comparison in detail.
Section 4.4 defines the process by which we evaluate the algorithms in order to assess the quality of each algorithm.

\section{Motivation}

\section{Algorithms}
This section gives a short introduction to the conceptual framework of supervised learning and then examines the algorithms included in our comparison. For further research, the interested reader is referred to \cite{Bishop} and \cite{SLFoundations}.

Supervised learning models receive a finite sequence 
$S = \{(x_1, y_1), (x_2,y_2), (x_3,y_3)\}$ from pairs of $X \times Y$ as inputs where $x \in X$ is a \textbf{feature} (also called input or observation) and $y \in Y$ is its corresponding \textbf{label} (also called output or target) and use a function $f : X \to Y$ that predicts any $y \in Y$ for an observation $x \in X$


The accuracy of a function is measured with \textbf{loss function} $L: Y \times Y \to \mathbb{R}$. $x_i$ that is passed to $h(x_i)$ from $y_i$ by which we can measure how accurate the predictions of a model are. $h_{opt}$ is then used to predict any target $h_{opt}(x_i)$ based on $x_i$.  

Predictions can generally be made for either a \textbf{regression} or \textbf{classification} problem, which differ in the nature of their target variables. While regression is used to predict continuous targets, classification finds its use in the prediction of discrete targets. The arrival time estimation problem is a regression task since we are estimating (continuous) arrival times.
Although various supervised learning models use different mathematical procedures to predict targets, all of them follow the principle of induction, meaning that general rules are inductively inferred from the observations. This is also called \textbf{generalization}.  
In order for a model to generalize the data properly, two things have to be avoided: \textbf{Underfitting} and \textbf{Overfitting}.


\subsection{Linear Models}

\begin{equation*}
	\hat y(x,w) = \sum_{i=0}^N w_ix_i
\end{equation*}
\subsection{Ensemble Learning}

\subsection{Neural Networks}
- Example
- Formal Definition

\section{Feature Selection}

- Raw Data 
- Manual feature selection
- 
- Feature learning via deep autoencoder

\section{Evaluation} 
- Use MSE, MAE, MAPE etc. to derive infos about accuracy
- Hyperparameter sensitivity analysis (Random Search) for robustness
- Variations in data set for robustness
--> Noise einführen 
--> Anzahl der Trainingsdaten
--> Andere Features
--> Runtime
\chapter{Computational Study}

\section{Data description}

\section{Parametrization of Methods}

\section{Results}








\chapter{Conclusion and Future Work}\label{chap:conc}
This paper compares different solutions for the problem of estimating arrival times in delivery routing from different points of view. Since the communication of accurate arrival times is crucial for customer satisfaction, arrival time estimation integrated in meal delivery routing is indeed important as we have detailed in the problem formulation. Therefore, we have decided to analyze different supervised learning methods that were present in prior related research from different viewpoints. The models included in our comparison were Linear Regression, GBDT and Random Forest. Complementary to that, we conducted our experiments on different data models that mainly differ in their dimensionality and thus in the amount of information they contain. When considered for a certain experiment, the respective supervised learning model was trained upon both data models each. Our experimental procedure begins with finding the sample sizes sufficient to train each model. Then, we conducted HPO for GBDT and Random Forest to analyze how different parameter configurations impact the prediction quality and which parameters that were considered for optimization are more or less important. Lastly, we tested the robustness of each model by inducing different levels of noise into our datasets.

All in all, we conclude that the models consistently yielded better results and were overall more robust when trained upon the raw data model proposed in this paper instead upon the crafted data model proposed by \cite{Hildebrandt2020_EAT} across all experiments, and thus could surpass state-of-the-art results. However, it has to be emphasized that former data model is significantly harder to maintain and to interpret due to the sheer amount of features it contains. That makes the raw data model far less scalable. All of the models however vastly outperformed PoM. Out of all the included models, GBDT performed the best overall followed by Random Forest.  Surprisingly, it could yield better results under higher levels of noise on the raw data. The hyperparameter optimization and analysis also gave great insight and intuition of how the considered models perform with different configurations. Although we derived which parameter configurations work better and which do not, the reader is reminded that the optimal parametrizations we presented are highly specific to the problem. Future work could consider estimating arrival times via deep learning or additionally use deep autoencoders to learn useful features and project them into a low-dimensional feature space. Another idea would be to conduct experiments in a similar fashion on different arrival time estimation problem settings. 

%%% ------------- Ende Textteil -------------

%%% Anhang
\bibliography{references}%%% obligatorisch         
%\anhang                 %%% optional, in der Datei anhang.tex

\end{document}
