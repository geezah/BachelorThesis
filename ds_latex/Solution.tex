

\chapter{Methodology}
This chapter presents our methodogolical approach and describes its elements in detail.
Section 4.1 gives an overview of the methodological approach used to evaluate and compare the models. 
Section 4.2 gives an introduction to supervised learning and presents the different models considered in the comparison in detail.
Section 4.3 describes how we will optimize model hyperparameters. 
Section 4.4 focusses on feature selection techniques applied in our computational study.

\section{Design of Experiments}

The primary goal of this study is to find the most suitable model for the given arrival time estimation problem. To attain the best possible solution to this problem, it is necessary to examinate different alternatives since the one model that does the best job on any problem does not exist as shown in the literature review. While \cite{Zhu2020_OFCTE_DL} use deep learning as a means of estimating arrival respectively delivery times for restaurant meal delivery, \cite{Hildebrandt2020_EAT} turn to GBDTs on a quite similar problem setting, whereas \cite{Liu2018_LM_PLM} concluded that linear models are the better choice overall. For that reason, we will conduct experiments that will allow us to analyze the behaviour of the involved models.
Further challenge lies in the selection of a well suited data model for the given problem which is highly non-trivial. 
As we will show later on, feature selection plays a crucial role for model performance.

\section{Algorithms}

This section shortly introduces the statistical framework of supervised learning and then proceeds with the detailed examination of how each approach and algorithm included in our comparison, namely linear models, tree-based ensembles and neural networks, function. 

\subsection{Framework}
As inputs, supervised learning algorithms receive \textbf{training data} in form of a finite set $ S = \{({x}_{2}, y_2), ({x}_{2}, y_2), \dots, ({x}_{n}, y_n)\}$ of N pairs from $ \mathcal{X} \times \mathcal{Y} $ where $ x_i $ is a \textbf{sample} described by $ p $ features, and is associated to it's corresponding \textbf{label} $ y_i $.
They return a \textbf{hypothesis} $ h: \mathcal{X} \to \mathcal{Y} $ that aims to predict the corresponding label $ y \in Y $ for any $ x \in X $, but especially for unseen samples or i.e. \textbf{test data} $ x \notin S $.
We further assume a joint probability distribution $ P $ over $ \mathcal{X} $ and $ \mathcal{Y} $ each pair is identically and independently distributed according to. 
Thus, $ h(x) $ can be treated as a random variable being conditionally distributed with $ P(y | x) $ for a given $ x $, and not as a deterministic function of $ x $. 
The prediction accuracy of $ h $ is measured with a  \textbf{loss function} $ L : Y \times Y \to \mathbb{R}^{\geq 0}$.
For a given pair $ ({x}_i, y_i) $, the loss function $ L(y_i, \hat{y}_i) $ represents how far a prediction $ \hat{y_i} $ for the i-th sample $ x_i $ is away from the actual corresponding label $ y_i $. 
The average loss across all $ (x,y) \in S $ is called \textbf{empirical risk}.
Thus, the goal of a supervised learning algorithm is to find the optimal hypothesis $ h^* $ for which the overall empirical risk is minimal: 
\begin{equation}
	h^* = \argmin_{h \in H} \dfrac{1}{n} \sum_{i=1}^{n} L(h(x_i), y_i).
\end{equation}

Predictions can generally be made for either \textbf{regression} or \textbf{classification} problems which differ in the nature of their target variable. 
While regression is used to predict continuous targets $ Y = \mathbb{R} $, classification predicts discrete $ Y $. The arrival time estimation problem is a regression task since we are estimating (continuous) arrival times.
Although various supervised learning models use different mathematical procedures to predict targets, all of them follow the principle of induction, meaning that general rules are inductively inferred from training data. This is also called \textbf{generalization}.  
In order for a model to generalize well, two phenomena have to be avoided: \textbf{Underfitting} and \textbf{Overfitting}. 
Poor predictions on both the training and test data indicate that a model is underfitted. Very good prediction accuracy for training samples $ x \in S $, but poor accuracy on test samples $ x \notin S $ indicate that a model is overfitted.
In supervised learning, this dilemma is known as the \textbf{bias-variance tradeoff}. 
An overfitted model is characterized by rather being low in bias and high in variance, whereas the same goes vice versa for underfitted models. Models high in bias and low in variance are rather simple, but cannot capture complex mappings due to high bias, e.g. linear models. On the other hand, models low in bias and high in variance are able to capture complex patterns in the data which is the reason why they fit given training data $ S $ very well. The problem with models high in variance arises when predictions are performed on independent samples $ x \notin S $, because the better the model is fitted to $ S $, the higher becomes the variability of prediction performance on independent samples. 
The key challenge therefore lies in finding a model balancing this tradeoff well and thus having good and consistent prediction quality especially on test data.

For that reason, we consider diverse models covering the wide spectrum of model complexities to discover the suitable options for the EAT. 
 
\subsection{Linear Models}

Linear regression models assume that every $ y $ can be predicted with a linear combination of inputs, i.e. the training data $ S $. In the following, we will concisely explain how linear regression works. For further explanations, the reader is refered to \cite{friedman2001elements}.

Let $ \textbf{X}^{T} = (X_1, X_2, \dots, X_p) $ denote the input matrix containing $ p $ column vectors with N components each, and $ \hat{Y} = (y_1, y_2, \dots, y_n)$ the column vector with the corresponding real-valued outputs. A simple linear regression model is a linear combination of coefficients $ \mathbf{w} = (w_0, \dots, w_p) $ and column vectors $ X $:
\begin{equation}\label{eq:linearmodel}
h(X) = w_0 + \sum_{j=1}^{p} X_jw_j
\end{equation}
with coefficient $ w_0 $ representing the bias term. The feature vectors $ X \in \textbf{X}^{T} $ can be of any form and be arbitrarily transformed since linear models only assume linearity in their coefficients, and not in their inputs.
A common method of approximating the hypothesis in linear regression is \textit{least squares}, which seeks to find parameters that minimize the squared differences across all N pairs the training set $ S $:
\begin{equation}\label{eq:argminlinear}
\argmin_{w} \sum_{i=1}^{N}(y_i - w_0 - \sum_{j=1}^{p} x_{ij}w_j)^2. 
\end{equation}
The coefficients resulting from the least squares minimization in \ref{eq:argminlinear} are then employed in \ref{eq:linearmodel}, which can be used to predict labels for given samples. 
Linear models are the better choice when it is known that the ground truth mapping between the independent and dependent variables of our model is linear since the model assumes linearity, is easy to implement and fast to train. The flipside of the coin is that more complex patterns in the can not be fully captured by linear models since it is always assumed that the underlying mapping of the given problem is linear which might not be the case for more complex scenarios. 

We will use the linear regression algorithm implemented in \textit{Scikit-Learn}, a library providing numerous machine learning algorihms written in Python designed by \cite{scikit-learn}.

\subsection{Tree-based Ensembles}

As shown in the literature review, tree-based ensembles like GBDTs and Random Forests are popular in the field of arrival time estimation. Ensemble learning is motivated by the idea that \textit{base learners} (synonym to \textit{model} or \textit{hypothesis}) produce rather inaccurate results on their own, but form an accurate prediction model when they are combined. For tree-based ensembles, decision trees are used as base learners. 
Generally speaking, decision trees partition the feature space into distinct regions and then, in the case of regression, fit a real-valued prediction to each region. 
A single decision tree is given by
\begin{equation}
h(x) = \sum_{m=1}^{M} c_m I(x \in R_m)
\end{equation}
where $ c_m $ is the constant representing the prediction for all samples belonging to the m-th terminal region $ R_m $. Adopting the mean squared error as the loss metric, the predicted constant $ \hat{c}_m $ for each terminal region $ R_m $ is the average of all $ y $ for each $ x \in R_m $:
\begin{equation}
	\hat{c}_m =  \dfrac{1}{N_{R_m}}\sum_{x_i \in R_m}^{} y_i
\end{equation}
with $ N_{Rm} $ being the number of samples in terminal region $ R_m $. Given this, the decision tree algorithm needs to find optimal binary splits. Solving this analytically is computationally infeasible. Therefore, decision trees are constructed in a greedy fashion.
Let $ j $ be the split variable, representing the feature based on which the split of the feature space is done on, and $ s $ its associated split threshold.
The initial partition creates two disjoint regions
\begin{equation}
	R_1 (j,s) = \{X | X_j < s \}, R_{2} = \{X | X_j > s\}.
\end{equation}
The algorithm seeks optimal $ j $ and $ s $ by finding constants that minimize the overall loss, which in our case is measured with the mean squared error, for both terminal regions $ R_1 $ and $ R_2 $:
\begin{equation}\label{minjs}
	\min_{j,s} \bigg [\min_{\hat{c}_1} \sum_{x_i \in R_1(j,s)} \dfrac{1}{2}(y_i - \hat{c}_{2})^2 + \min_{\hat{c}_2} \sum_{x_i \in R_{2}(j,s)} \dfrac{1}{2}(y_i - \hat{c}_{2})^2\bigg]
\end{equation}
Having found the optimal split variable and split threshold, this procedure is repeated for each terminal region until a termination condition holds, e.g. a fixed maximal depth for the tree or a lower bound wrt. number of samples in each terminal region. 
\newline
\newline
TODO: Shortcomings of decision trees
\newline
\newline
Understanding the greedy approach used to build regression trees, we now can advance to tree-based ensemble learning techniques which basically combines several regression trees to make predictions each in their own way. 
The two ensemble learning approaches relevant for our work are \textbf{bagging}, shorthand for \textit{bootstrap aggregating}, and \textbf{boosting}. 
Algorithms belonging to bagging first create bootstrap data sets $ S^{*} $ by randomly drawing samples with replacement from training data S, fit each $ S^{*b} $ a hypothesis $ h^{*b}(x) $ and take the average across all hypotheses as the bagging estimate: 
\begin{equation}\label{average}
	h_{bag}(x) = \dfrac{1}{B} \sum_{b=1}^{B} h^{*b}(x).
\end{equation}
Since decision trees are low in bias and high in variance due to their ability to capture complex mappings, bagging is motivated by the idea that averaging across all base learners benefits from the strong law of large numbers and therefore leads to a reduction in variance. 
Random Forests belong to the class of bagging algorithms and work as follows:
For an arbitrarily chosen and fixed amount of estimators $ B \in \mathbb{N}$, we create bootstrap data sets $ S^{*b} $ with $ b = 1, \dots, B $ of size $ N $ from the training data $ S $. The b-th regression tree $ T_b $ is grown to $ S^{*b} $ by first selecting a subset of $ m $ features with $ m \leq p $, and then growing a regression tree according to CART as described above.
Having grown each bootstrap data set $ S^{*b} $ its own regression tree $ T_b $, random forest for regression averages across all hypotheses of each regression tree by applying equation \ref{average} and returns
\begin{equation}
h^{B}_{rf}(x) = \dfrac{1}{B} \sum_{b=1}^{B} T_b(x)
\end{equation}
as the bagging estimate.
\newline
\newline
TODO: Introduce concept of boosting and GBDTs before diving deep into technical details.
\newline
\newline
The first step in GBDT is to initialize a model with a constant $ \hat{c} $ that minimizes the overall loss across all training samples.
\begin{equation}
h_{0}(x) =  \argmin_{\hat{c}} \sum_{i=1}^n L(y_i,\hat{c}) \label{gbdt_step1}
\end{equation}
Using mean squared error as the loss metric, the predictor minimizing the overall loss is simply the average of all $ y $.
Given the initial model $ h_0(x) $, M regression trees can now be build sequentially as shown in the following. 
The first step computes pseudo-residuals $ r_{im} $ between the prediction value of the former learner for the i-th sample $ h_{m-1}(x) $ and the actual i-th target for the m-th tree $ y_i $ by deriving the loss function w.r.t to $ h_{m-1}(x) $: 
\begin{equation}
	r_{im}\ = - \bigg[\dfrac{\delta L(y_i, h(x_i))}{\delta h(x_i)}\bigg]_{h(x) = h_{m-1}(x)}
\end{equation} 
for every i-th example.
In the second step, the algorithm fits a regression tree to every $ r_{im} $ and creates disjoint regions $ R_{jm} $ for $j = 1, ..., J_m$. 
In the third step, we compute: 
\begin{equation}
	\hat{c}_{jm} = \argmin_{\hat{c}} \sum_{x_i \in R_{ij}} L(y_i, h_{m-1}(x_i) + \hat{c})
\end{equation}
for every terminal region j in the m-th tree. 
Given this, we now update the former weak learner with the new one:
\begin{equation}
	h_m(x) = h_{m-1}(x) + \alpha \sum_{j=1}^{J_m} \hat{c}_{m}I(x \in R_{jm})
\end{equation}
where $ \alpha $ is the learning rate, and $ I(\cdot) $ is the indicator function.
Hence, we use the GBDT implemetation of \textit{lightGBM}.



\subsection{Neural Networks for Regression}\label{sec:nn}


\section{Feature Selection}

A significant role for the success of machine learning models in general plays the data we gather and how we make sense of it - i.e. how we turn raw data into informative features - since machine learning models can only learn from the data they receive for the learning process. Gerenally, we want to maximize the amount of informative features that predict our target comparably good while minimizing the amount of features we use at the same time. High-dimensional feature spaces are not desired as training time exponentially increases and prediction performance decreases.
Therefore, selecting and engineering informative features is not to be underestimated. 


