\chapter{Conclusion and Future Work}\label{chap:conc}
This paper compares different solutions for the problem of estimating arrival times in delivery routing from different points of view. Since the communication of accurate arrival times is crucial for customer satisfaction, arrival time estimation integrated in meal delivery routing is indeed important as we have detailed in the problem formulation. Therefore, we have decided to analyze different supervised learning methods that were present in prior related research from different viewpoints. The models included in our comparison were Linear Regression, GBDT and Random Forest. Complementary to that, we conducted our experiments on different data models that mainly differ in their dimensionality and thus in the amount of information they contain. When considered for a certain experiment, the respective supervised learning model was trained upon both data models each. Our experimental procedure begins with finding the sample sizes sufficient to train each model. Then, we conducted HPO for GBDT and Random Forest to analyze how different parameter configurations impact the prediction quality and which parameters that were considered for optimization are more or less important. Lastly, we tested the robustness of each model by inducing different levels of noise into our datasets.

All in all, we conclude that the models consistently yielded better results and were overall more robust when trained upon the raw data model proposed in this paper instead upon the crafted data model proposed by \cite{Hildebrandt2020_EAT} across all experiments, and thus could surpass state-of-the-art results. However, it has to be emphasized that former data model is significantly harder to maintain and to interpret due to the sheer amount of features it contains. That makes the raw data model far less scalable. All of the models however vastly outperformed PoM. Out of all the included models, GBDT performed the best overall followed by Random Forest.  Surprisingly, it could yield better results under higher levels of noise on the raw data. The hyperparameter optimization and analysis also gave great insight and intuition of how the considered models perform with different configurations. Although we derived which parameter configurations work better and which do not, the reader is reminded that the optimal parametrizations we presented are highly specific to the problem. Future work could consider estimating arrival times via deep learning or additionally use deep autoencoders to learn useful features and project them into a low-dimensional feature space. Another idea would be to conduct experiments in a similar fashion on different arrival time estimation problem settings. 