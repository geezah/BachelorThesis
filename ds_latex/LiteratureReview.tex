\chapter{Literature Review}

In this section, the paper presents related literature. 


\section{Most related work}

This work is mostly inspired by \cite{Hildebrandt2020_EAT}, who contributed a supervised offline and another supervised online-offline approach to predict arrival times in the Restaurant Meal Delivery problem setting with estimation of arrival time (RMDPEAT), a dynamic pick up and delivery problem with uncertainty in travel times, processing times and order bundling firstly  presented in \cite{UlmerRMDP}. A more detailed problem explanation of the RMDPEAT is done in Section 3. 
In their offline approach, \cite{Hildebrandt2020_EAT} map spatial, temporal, routing, and processing features based on the RMDPEAT to expected arrival times by means of a gradient boosting decision tree (GBDT) model. In their online-offline approach, they calculate the running arrival time average online by training a supervised neural network that approximates the decisions a given routing policy would make for a given state.
This paper can be seen as complementary to their paper since we aim to estimate arrival times based on the same underlying problem setting via several supervised learning algorithms, including GBDTs. However, due to hardware limitations, partial and full online approaches are excluded. 

%\cite{Zhu2020_OFCTE_DL} predict arrival times by means of deep learning for a vehicle routing problem where requests come in dynamically. Uncertainty is present in bundling, courier travel times, courier waiting times at restaurants and cooking times. Additional information 

\section{Arrival Time Estimation}

%\cite{UlmerBarrett2017_TWAP} estimate mean arrival times for a vehicle routing problem with stochastic requests (VRPSR) in home-attended delivery using value function approximation. While their model assumes uncertainty in travel 



%\cite{Liu2018_LM_PLM} ETA x Machine Learning x RMDP
 
   


%\cite{Zhang2013} formulate a SVRPTW with soft-time window constraints and stochastic travel and service times. Their objective function seeks to minimize vehicle employing costs, travel times and deviances w.r.t expected arrival times which are determined by the sum of expected service and travel times. The expected arrival times are then used to calculate a minimum on-time probability which every vehicle must fulfill for every visited customer.  


%\cite{Vanajakshi2007_SVR}
%\cite{Masiero2011_SVR}

\cite{Chen2004_ANNKalman} who proposed an artifical neural network.

%\cite{Wang2018_WDR_DL} propose an offline and online method both based on a Wide Deep Recurrent Neural Network (WDRNN) model to predict vehicle travel times for single origin-destination routes by aggregating spatial, temporal, traffic, personalized and augmented features. In their offline comparison, they compared their approach to competing classical machine learning methods and route-based ETA, latter being the sum of weighted average travel times of subsegments in the route. Two indications of their results are interesting in our viewpoint: First, all machine learning methods included in their experiment outperformed route-based ETA. Secondly, their deep learning approach outperformed the competing classical machine learning methods.  

A significant amount of work where travel times are estimated via supervised learning has been done in the field of intelligent transportation systems for single origin-destination problems. 

To predict travel times on freeway routes for different short-term forecasting horizons, \cite{Vanajakshi2007} use support vector regression (SVR) based on estimated historical route travel times and concluded that SVR and ANNs perform comparably well. 
\cite{Siripanpornchana2016_AnnWithDbnFS} propose a neural network architecture consisting of a deep belief network which learns features in an unsupervised fashion based on historical route travel time information it receives as its input, and a sigmoid regression layer that predicts travel times based on these learned features. \cite{Cheng2019_GBDT} use a GBDT model based on manually selected travel time features and traffic state related variables like occupancy, speed, number of vehicles etc. \cite{Cheng2019_GBDT} found that GBDTs outperform feedforward neural networks and support vector machines.

\cite{jindal2017unified} estimate taxi travel times with a unified approach based on raw NYC taxi data. They concatenate two neural networks, where the first one uses spatial features to predict travel distances, and the second one uses these predicted distances and additional temporal information to predict travel times. 
In contrast to them, \cite{Huang2018_GBDT} and \cite{huang2020travel_GBDT} compare several tree-based learning methods to predict travel times for NYC taxi data on different horizons each, among them random forests (both), GBDTs (both), and CART (only \cite{huang2020travel_GBDT}). While \cite{Huang2018_GBDT} selected features by means of principal component analysis, \cite{huang2020travel_GBDT} engineered them manually. Both ended up using spatial and temporal features mainly. Their results indicate that all tree-based ensemble methods are able to predict travel times more accurately than the respective benchmark algorithms (CART and naive approach in \cite{huang2020travel_GBDT}; linear and logistic regression in \cite{Huang2018_GBDT}).



