\chapter{Literature Review}

This chapter presents prior research on offline arrival time estimation via supervised learning. The goal is to give the reader an overview about the   


\section{Most related work}
This work is mostly inspired by \cite{Hildebrandt2020_EAT}, who contributed a supervised offline and online-offline approach to predict arrival times in the Restaurant Meal Delivery problem setting, a dynamic pick up and delivery problem with uncertainty in travel times, processing times and order bundling originally  presented in \cite{UlmerRMDP}.
In their offline approach, \cite{Hildebrandt2020_EAT} map spatial, temporal, routing, and processing features based on the RMDPEAT to expected arrival times by means of a gradient boosting decision tree (GBDT) model. This paper can be seen as complementary to their paper since we aim to estimate arrival times based on the same underlying problem setting via several supervised learning algorithms, including GBDTs. Due to hardware limitations, partial and full online approaches are excluded from this study. 

Other than \cite{Hildebrandt2020_EAT}, research on arrival time estimation via supervised learning in dynamic pick-up and delivery settings is very limited. To the best of the authors knowledge, \cite{Zhu2020_OFCTE_DL} and \cite{Liu2018_LM_PLM} are the only works that fit this description. 

\cite{Zhu2020_OFCTE_DL} predict arrival times by means of deep learning for a vehicle routing problem where requests come in dynamically. Uncertainty is present in bundling, courier travel times, courier waiting times at restaurants and cooking times. Besides using temporal, spatial and processing features for travel time prediction, they additionally include dish specific features and weather conditions. In contrast to \cite{Hildebrandt2020_EAT}, they include no routing information. They instead introduce a separate component that ranks courier assignments w.r.t. logistics cost and customer inconvenience. Their analysis indicates that their proposed deep learning architecture, inter alia, outperforms a GBDT approach w.r.t. prediction accuracy.

\cite{Liu2018_LM_PLM} compare linear and piece-wise linear prediction models for travel time estimation based on spatial, temporal and order-related features, and integrate travel time prediction into the order assignment problem with uncertainty in bundling, travel times and service times. The order assignment problem aims to assign orders in a way that the assignments minimize the total delivery delay over all driver routes. Analyzing the prediction models w.r.t to their accuracy, tractability and interpretability, they found that random forests yield the most accurate results but are computationally less tractable due to exponential runtimes and interpretable than linear models. Amongst the linear models, lasso regression obtained the smallest mean squared test error.


\section{Arrival Time Estimation}


%\cite{Chen2004_ANNKalman} 

%\cite{Wang2018_WDR_DL} propose an offline and online method both based on a Wide Deep Recurrent Neural Network (WDRNN) model to predict vehicle travel times for single origin-destination routes by aggregating spatial, temporal, traffic, personalized and augmented features. In their offline comparison, they compared their approach to competing classical machine learning methods and route-based ETA, latter being the sum of weighted average travel times of subsegments in the route. Two indications of their results are interesting in our viewpoint: First, all machine learning methods included in their experiment outperformed route-based ETA. Secondly, their deep learning approach outperformed the competing classical machine learning methods. 

With this section, we broaden the scope from arrival time estimations via supervised learning for dynamic pick-up and delivery problems to arrival time estimations via supervised learning for vehicle routes in general.
Tab. 1 classifies the literature on arrival time estimation for vehicle routes with regards to the problem setting and the solution approach. 
By \textit{Route type}, the table identifies if arrival times are estimated for either single origin-destination pairs or sequences of them. 
By \textit{Uncertainties}, the table refers to uncertain elements in the underlying problem settings. Uncertainty in request indicates that customer requests are not certainly known at the start of the problem and arrive dynamically. Uncertainty in travel and service times expresses itself through uncertain weather conditions, traffic congestion and individual challenges when serving customers (e.g. parking or waiting times). Uncertainty in processing occurs when  

For single origin-destination problems, significant amount of work where travel times are estimated via supervised learning has been done for intelligent transportation systems. 

To predict travel times on freeways for different short-term forecasting horizons, \cite{Vanajakshi2007} use support vector regression (SVR) based on estimated route travel times from prior research and conclude that SVR and ANNs perform comparably good and outperform prediction methods which are based on historical average values or most recently obtained travel time information. 
\cite{Siripanpornchana2016_AnnWithDbnFS} propose a deep learning architecture consisting of a deep belief network and a sigmoid regression layer. Former learns features in an unsupervised fashion based on historical route travel times as inputs, latter then estimates travel times based on these learned features. \cite{Cheng2019_GBDT} use a GBDT model based on manually selected travel time features and traffic state related variables. In their study, GBDTs outperform feedforward neural networks and support vector machines.

For taxi travel time prediction, \cite{jindal2017unified} propose a unified approach based on raw NYC taxi data. They concatenate two neural networks, where the first one uses spatial features to predict travel distances, and the second one uses these predicted distances and additional temporal information to predict travel times. Although they solely compared their approach to other deep learning architectures.
In contrast to them, \cite{Huang2018_GBDT} and \cite{huang2020travel_GBDT} compare several tree-based learning methods to predict travel times on different horizons each based on NYC taxi data as well, among them random forests (both), GBDTs (both), and CART (only \cite{huang2020travel_GBDT}). While \cite{Huang2018_GBDT} selected features by means of principal component analysis, \cite{huang2020travel_GBDT} engineered them manually. Both ended up using spatial and temporal features mainly. Their results indicate that all tree-based ensemble methods are able to predict travel times more accurately than the respective benchmark algorithms (CART and naive approach in \cite{huang2020travel_GBDT}; linear and logistic regression in \cite{Huang2018_GBDT}).


