\chapter{Introduction}
%%% Motiviere Problem
Nobody likes waiting for food - or waiting at all. In fact, when doing research on this topic, psychologists found out that increased waiting times generally have a significant negative impact on customer satisfaction and loyalty (\citealt{WaitingTime1}, \citealt{WaitingTime2}). Combine this with the fact that the online food delivery market is in high demand as more than 700 million people globally used food delivery services in 2017 and twice as many expected in 2024 \citep{Statista1}. In the US, the biggest platform-to-delivery companies like GrubHub, Domino's Pizza and UberEATS have close market shares. Thus, waiting not only becomes a large scaled economic issue due to the already very big user base. The participants on the provider side, be it the numerous restaurants or the meal delivery platforms, also operate in highly competitive environments. However, one might not expect that the customers’ own perceptions regarding their waiting time negatively affect the perceived service quality stronger than actual waiting times do (\citealt{waiting5}, \citealt{waiting6}). Given these points, we can conclude that a key challenge, but also a chance lies in the communication of accurate arrival time estimations to customers. 
%%% Motiviere Lösung

A popular and common way to tackle prediction tasks in general is the use of machine learning techniques. In the last years, it has proven to be a powerful forecasting tool in a wide variety of settings. For arrival time estimation, a go-to choice seems to be the use of supervised learning: Exemplary, Hildebrandt and Ulmer use Gradient Boosting Decision Trees in their supervised learning approach to map state features directly to expected arrival times in food delivery. However, different researchers use different approaches on similar problem settings: In contrast to \cite{Hildebrandt2020_EAT}, \cite{Zhu2020_OFCTE_DL} use deep learning to predict accurate order fulfillment cycle times in a meal delivery environment. Moreover, both propose different data models their proposed supervised model learns upon. This is not an isolated case. As we will show later on, such instances are rather mounting up. 
To yield optimal results, not only the choice of the right data and the right learning model is crucial, but moreover a suitably chosen parametrization of the learning model as well. The art of optimizing model parameters is a highly active research area as hyperparameter has proven that it can improve the performance of models in different environments (\citealt{HPOMotivation}, \citealt{WU201926}).
Drawing the connection to the highly competetive market that online food delivery certainly is, we can conclude that choosing a model that yields the best possible results is a significant competitive advantage for any meal delivery. For that reason, we intend to conduct further research in this area.

This paper examines the forecast quality of different offline supervised learning models for arrival time estimation problems in meal delivery from different view points. It builds up on the work of \cite{Hildebrandt2020_EAT}, who formulate the \textit{Restaurant Meal Delivery Problem with Arrival Times}. The RMDPEAT combines arrival time estimation with delivery routing. The underlying vehicle routing setting for the RMDPEAT originates from the \textit{Restaurant Meal Delivery Problem}, a dynamic pick-up and delivery problem originally presented in \cite{UlmerRMDP}. We operate on the same problem setting as \cite{Hildebrandt2020_EAT} and will therefore also operate on similar data. The dataset contains historical meal delivery data collected within Iowa City.  

This paper is organized as follows. Firstly, we will review and discuss literature that focuses on arrival time estimations done via offline supervised learning methods in chapter \ref{chap:review}. We will then proceed with the problem formulation in \ref{chap:prob}. There, we will recap the RMDP originally presented in \cite{UlmerRMDP}, and work out the differences to the RMDPEAT presented in \cite{Hildebrandt2020_EAT}. By that, we intend to motivate the integration and use of arrival time estimation in the RMDP. In chapter \ref{chap:method}, we first present the design of our experimental pipeline consisting of three different experiments which can be understood as the main conrtibutions of this paper:
\begin{enumerate}
	\item Analyze the model performance behaviour for subsets of different size in order to capture the sufficient sample size needed to train the respective model.
	\item Conduct hyperparameter optimization to attain the optimal configurations and further analyze in what way the parameters of the model impact the outcome. 
	\item Examine the robustness of the models by training the models with different levels of noise in the data.
\end{enumerate}
Then, we present and explain our data model selection, and last but not least work out the supervised learning methods used for arrival time prediction in detail. We will then advance to the computational study in chapter \ref{chap:comp}, where we present and discuss the results of our experiments.
Chapter \ref{chap:conc} draws the conclusions and gives outlook on future work.
 