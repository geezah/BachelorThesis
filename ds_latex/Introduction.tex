\chapter{Introduction}

Nobody likes waiting - be it when you order your next book on Amazon for class or the pizzas at UberEATS for your birthday. In fact, when doing research on this topic, psychologists found out that increased waiting times generally have a significant negative impact on customer satisfaction and loyalty (\citealt{WaitingTime1}, \citealt{WaitingTime2}). If we take a glance at the online food delivery market, there are more than 700 million people globally that used food delivery services in 2017 with twice as many users being expected in 2024 \citep{Statista1}. According to forecasts for the same time frame, the userbase of eCommerce platforms in general will grow from 2.480 billion to roughly 4.6 billion users [4]. Thus, waiting becomes a large scaled economic issue. What one might not expect is that the customers’ own perceptions regarding their waiting time negatively affect the perceived service quality stronger than actual waiting times do [5,6]. Given these points, we can conclude that a key challenge lies in the communication of accurate arrival time estimations to customers. 

Current research tackles this task commonly by means of machine learning techniques. Exemplary, Hildebrandt and Ulmer used Gradient Boosting Decision Trees in their offline approach to map state features directly to expected arrival times in food delivery [7]. Zhu et al. used deep learning to predict accurate Order Fulfillment Cycle Times [8]. Hardly any kind of supervised learning model comparison for arrival time estimation purposes in vehicle routing settings can be found. Therefore, we intend to conduct further research in this area.  

This paper examines the forecast quality and performance of different supervised machine learning models for arrival time estimation problems (ETA) in meal delivery. To accomplish this task, the algorithms will be trained on historical meal delivery data collected within Iowa City. 

This paper is organized as follows. Firstly, we will review and discuss literature that focuses on arrival time estimation via offline supervised learning. We will then proceed with the problem formulation, where we problem generally and define the model with respect to the underlying problem of our experimental design. Several steps are taken in the solution approach that follows upon our model definition. In the beginning, we give an introduction to fundamental concepts in supervised learning and present techniques used at different stages in the machine learning pipeline generally for the sake of understanding the way each of them functions. We will then advance to the computational study, where we present our experimental design and discuss the results. Finally, we will create variations of our experimental design in order to analyze each algorithm’s robustness and impact in these different settings.