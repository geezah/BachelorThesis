\chapter{Problem formulation}

This chapter gives a short overview of the classical RMDP from \cite{UlmerRMDP} and the RMDPEAT introduced in \cite{Hildebrandt2020_EAT}. Then, we give a formal, but concise description of the EAT.
\section{Restaurant Meal Delivery Problem}

According to the taxonomy for dynamic vehicle routing problems introduced in \cite{psaraftis}, the classical RMDP belongs to the category of dynamic (1) and stochastic (2) problems since (1) drivers routes need to be reoptimized over time by means of a routing policy and (2) only the probability distributions of inputs are known at the start of the problem respectively. In the RMDP, customers are distributed across a known service area and request service during a finite service time window. Drivers are independent contractors distributed in the service area and are dispatched by the meal delivery platform that provides meal delivery service for participating restaurants. The RMDP process is triggered with the arrival of customer requests. When a customer requests service on the meal delivery platform, the assignment policy proposed in \cite{UlmerRMDP} assigns the customer an available driver. When this is done, the dispatcher updates the routes of drivers. It is therefore assumed that the routing policy is exogenously given. The assigned driver then drives to the restaurant, picks the meal up and eventually waits at the restaurant if meals are not ready, and then delivers it to the customer. This whole process is impacted by uncertainty in the arrival of customer requests, pick-up and delivery times of drivers, and meal preparation times of restaurants. Further, The RMDP allows order bundling. Order bundling refers to the possibility of assigning a vehicle more than one order at a time which results in greater flexibility for assignment decisions.  
Overall, the RMDP seeks maximization in the amount of served customers and minimization in delivery delays.
 
\section{Restaurant Meal Delivery Problem with Arrival Time Estimation}

The classical RMDP assumes that planned arrival times are the result of assump-tions made by the dispatcher. In contrast to \citet{UlmerRMDP}, \citet{Hildebrandt2020_EAT} account for arrival time estimations in their problem by providing customers requesting service on the platform estimated arrival times for each restau-rant. That way, customers are given the choice to either order from a prefered restaurant if the arrival time preferences are met, or do not order at all. As we already pointed out in the very beginning, this is crucial as there is evidence of arrival delay having a negative impact on customer satisfaction. 

\section{Estimating arrival times for delivery routing}

\cite{Hildebrandt2020_EAT} define the RMDPEAT as a dynamic decision process. Based on that, they further formulate the problem of estimating arrival times for delivery routing (EAT), which is just the EAT part of the RMDPEAT process, formally and in detail as well. In the following, we recap their EAT formulation formally since we operate on the same underlying DVR problem setting and share a the same task in predicting arrival times, but keep our explanations concise. For a in-depth explanation of the EAT, the interested reader is referred to section 3.2 in \cite{Hildebrandt2020_EAT}. 

Let $ c_k $ denote the k-th customer requesting service at time point $ t_k $, $ \mathcal{N}_k $ the set of customers that have requested service but have not been served yet, $ R_k $ the set of restaurants, and $ \Theta_k $ the set of planned routes for each driver at decision epoch $ k $. The decision state for the RMDPEAT is therefore given as $ S_k = (t_k, \mathcal{N}_k \cup c_k, \mathcal{R}_k, \Theta_k) $. Decisions are triggered when customers request service and happen in two stages: First, potential routing decisions $ \Theta_{ik} $ for every restaurant $ i = 1, \dots, |\mathcal{R}_k| $ the customer can order from are computed. The union of the RMDPEAT state $ S_k $ with these potential routing decisions $ \Theta_{ik} $ represents the EAT state $ S^{EAT}_k := (S_k \cup \Theta_{ik})$. Based on $ S^{EAT}_k $, we estimate arrival times for each potential routes $ \Theta_{ik} $ respectively and denote them as $ X(S^{EAT}_k) := (X_{ik})_{i = 1, \dots, |\mathcal{R}_k|}$. Customers then choose a prefered restaurant based on their individual preference functions $ \Phi_k((X_{ik})_{i \in \{1,\dots, R_k\}}) $.  
The objective of the EAT is defined as
\begin{equation}\label{equation:3.1}
	\min_{X(S^{EAT}_k)(j) \in \mathbb{R}_+} 
	\mathbb{E}_{S^{EAT}_{k}} 
	(|| A(S^{EAT}_{k})(j) - X(S^{EAT}_{k})(j)||^{2}_{2}).
\end{equation}
Equation \ref{equation:3.1} seeks to minimize the expected deviance of the estimated arrival time $ X(S^{EAT}_{k})(j) $ from the actual arrival time $ A(S^{EAT}_{k})(j) $ where $ j $ represents the restau-rant that the customer will chose. The deviance is measured with the mean squared error. 

%A decision epoch $ k $ occurs whenever a customer requests arrival times, in other words requests delivery. 
%State $ S_k = (t_k, \mathcal{N}_k \cup c_k, \mathcal{R}_k, \Theta_k)$ contains the time point of the request $ t_k $, the set of customers $ \mathcal{N_k} $ that have requested service but have not been served yet plus the customer that requests service at time $ t_k $, the restaurants and their workloads $ \mathcal{R}_k $ described by a location, the queue of requested meals to prepare and the preparation start time, and the set of planned routes $ \Theta_k $ given by a sequence of pickup and delivery actions which in turn are given by an origin, destination, the time when the pickup or delivery action is started and the duration of the action. 
%A decision determines the set of planned routes $ \Theta_{ik} $ first and then estimates arrival times $ X_{ik} := X (S_k \cup \Theta_{ik}) \in \mathbb{R}$ for each restaurant $ i = 1, \dots, |\mathcal{R}_k| $. 
%Transition of state $ S_k $ to the state at the next decision epoch $ S_{k+1} $ is then realized when a routing decision is made and the customer $ c_k $ makes his restaurant choice at time $ t_k $. 
%State transitions result in updated planned routes $ \Theta_{k+1} $ given by $ \Theta_k $ ex pickup and delivery actions carried out between $ t_k $ and $ t_{k+1} $. 
%Following sources of uncertainty impact the RMDPEAT process: 
%(1) Customers are unknown until they request service, (2) meal preparation times of restaurants, 
%(3) parking and delivery times of drivers, and 
%(4) the restaurant choice of customer $ c_k $ determined by an individual preference function $ \Phi_k((X_{ik})_{i \in \{1,\dots, R_k\}}) $ which receives a vector of corresponding arrival times for each restaurant as input, and returns the customers restaurant choice.
%Thus, the routing decision itself is uncertain due to (1), realization of pickup and delivery actions is uncertain due to (2) and (3), and restaurant workloads are uncertain due to (3) and (4). 
%$ \mathcal{N}_{k+1} $ depends on all four sources of uncertainty and the routing decision.
%Last but not least, the RMDP seeks to maximize the number of served customers within the service horizon and minimize delays of predicted arrival times wrt. actual arrival times. Latter is the focus of \cite{Hildebrandt2020_EAT} and therefore also the focus of this work.
    