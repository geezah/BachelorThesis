\chapter{Problem formulation}

This chapter gives a short overview of the RMDPEAT introduced in \cite{Hildebrandt2020_EAT} and then formulates the EAT component in detail. The RMDPEAT combines the dynamic pick-up and delivery problem presented in \cite{UlmerRMDP} with arrival time estimation and hence accounts for both processes.  
%The extension of the RMDP with EAT is important since customer preferences are taken into account. 
%For a detailed route-based Markov decision process-model of the RMDPEAT, the interested reader is referred to §A.1 in the Appendix.

\section{RMDPEAT} 


A decision epoch $ k $ occurs whenever a customer requests arrival times. 
State $ S_k = (t_k, c_k, \mathcal{N}_k, \mathcal{R}_k, \Theta_k)$ contains the time point of the request $ t_k $, the set of customers $ \mathcal{N_k} $ that have requested service but have not been served yet, the restaurants and their workloads $ \mathcal{R}_k $ described by a location, the queue of requested meals to prepare and the preparation start time, and the set of planned routes $ \Theta_k $ given by a sequence of pickup and delivery actions which in turn are given by an origin, destination, the time when the pickup or delivery action is started and the duration of the action. 
A decision determines the set of planned routes $ \Theta_{ik} $ first and then estimates arrival times $ X_{ik} := X (S_k \cup \Theta_{ik}) \in \mathbb{R}$ for each restaurant $ i = 1, \dots, |\mathcal{R}_k| $.
Transition of state $ S_k $ to the state at the next decision epoch $ S_{k+1} $ is realized when a routing decision is made and the customer $ c_k $ makes his restaurant choice at $ t_k $. State transitions result in updated planned routes $ \Theta_{k+1} $ given by $ \Theta_k $ ex pickup and delivery actions carried out between $ t_k $ and $ t_{k+1} $. Following sources of uncertainty impact the RMDPEAT process: (1) Customers are unknown until they request service, (2) meal preparation times of restaurants, (3) parking and delivery times of drivers, and (4) the restaurant choice of customer $ c_k $ determined by an individual preference function $ \Phi_k((X_{ik})_{i \in \{1,\dots, R_k\}}) $ which receives a vector of corresponding arrival times for each restaurant as input, and returns the customers restaurant choice. Thus, the routing decision itself is uncertain due to (1), realization of pickup and delivery actions is uncertain due to (2) and (3), and restaurant workloads are uncertain due to (3) and (4). $ \mathcal{N}_{k+1} $ depends on all four sources of uncertainty and the routing decision.
Last but not least, the RMDP seeks to maximize the number of served customers within the service horizon and minimize inaccuracy of arrival time estimation wrt. actual arrival times. Latter is the focus of \cite{Hildebrandt2020_EAT} and therefore also the focus of this work.
 
\section{Estimating arrival times for Restaurant Meal Delivery}

























































































    